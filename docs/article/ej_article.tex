\documentclass[spanish,a4paper,10pt]{article}

\usepackage{latexsym,amsfonts,amssymb,amstext,amsthm,float,amsmath}
\usepackage[spanish]{babel}
\usepackage[latin1]{inputenc}
\usepackage[dvips]{epsfig}
\usepackage{doc}



\begin{document}
\title{Imforme de n�mero $\pi$}
\author{Ana G�mez P�rez\\ Pr�ctica de Laboratorio 10}
\date{11 de abril de 2014}

\maketitle

\begin{abstract}
El objetivo de este documento es saber m�s sobre el n�mero $\pi$ y exponerlo en un pdf con \LaTeX{}~\cite{LaTeXpg} en la universidad (figura)~\ref{et:graf} .
\end{abstract}

%++++++++++++++++++++++++++++++++++++++++++++++++++++++++++++++++++++++
\section{historia del n�mero $\pi$}
$\pi${}~\cite{PIpg} es un n�mero irracional, cociente entre la longitud de la circunferencia y la longitud
de su di�metro. Se emplea frecuentemente en matem�ticas, f�sica e ingerier�a. El valor num�rico
de $\pi$ truncado a sus diez primeras posiciones decimales, es el siguiente: 3,1415926535...

 \begin{figure}[t]
\begin{center}
\includegraphics [scale=0.075]{imagen1.eps}
\label{et:graf}
\end{center}
\end{figure}

%++++++++++++++++++++++++++++++++++++++++++++++++++++++++++++++++++++++
\section{Historia del valor $\pi$}
La b�squeda del mayor n�mero de decimales de n�mero $\pi$ ha supuesto un esfuerzo constante de numerosos cient�ficos a lo largo de la historia.
Algunas aproximaciones hist�ricas de $\pi$ son las siguientes:
 \subsection{Antiguo Egipcio}
 \subsection{Antigua Babil�nia}



Algunas aproximaciones hist�ricas (tabla)~\ref{eq:Tabla} de valores de $\pi$, anteriores a la �poca computacional\footnote{a partir del siglo 80}, se muestran en la siguente tabla:\\ 
\begin{table}{}
\caption{Tabla del n�mero}
\begin{center}
\begin{tabular}{|l|r|c|}
Cultura    &  Aproximacion & Error\\ \hline
Egipcia    &  3,1605       & 6016 ppm \\ \hline
Babil�nica &  3,125        & 5282 ppm\\ \hline
India      &  3,09         & 16422 ppm\\ \hline
\label{eq:Tabla}
\end{tabular}
\end{center}
\end {table}
\begin{enumerate}
  \item
\end{enumerate}

%++++++++++++++++++++++++++++++++++++++++++++++++++++++++++++++++++++++++++

\addcontentsline{toc}{chapter}{Bibliograf�a}
\bibliographystyle{plain}


\bibliography{ej_article}
\nocite{*}

\end{document}
