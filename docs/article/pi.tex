\documentclass[spanish,a4paper,10pt]{article}

\usepackage{latexsym,amsfonts,amssymb,amstext,amsthm,float,amsmath}
\usepackage[spanish]{babel}
\usepackage[latin1]{inputenc}
\usepackage[dvips]{epsfig}
\usepackage{doc}

%%%%%%%%%%%%%%%%%%%%%%%%%%%%%%%%%%%%%%%%%%%%%%%%%%%%%%%%%%%%%%%%%%%%%%
%123456789012345678901234567890123456789012345678901234567890123456789
%%%%%%%%%%%%%%%%%%%%%%%%%%%%%%%%%%%%%%%%%%%%%%%%%%%%%%%%%%%%%%%%%%%%%%
%\textheight    29cm
%\textwidth     15cm
%\topmargin     -4cm
%\oddsidemargin  5mm

%%%%%%%%%%%%%%%%%%%%%%%%%%%%%%%%%%%%%%%%%%%%%%%%%%%%%%%%%%%%%%%%%%%%%%
\begin{figure}[t]
\begin{center}
\includegraphics [scale=0.075]{imagen1.eps}
\end{center}
\end{figure}

\begin{document}
\title{Imforme de n�mero $\pi$}
\author{Ana G�mez P�rez\\ Pr�ctica de Laboratorio \#10}
\date{11 de abril de 2014}

\maketitle

\begin{abstract}
El objetivo de este documento es saber m�s sobre el n�mero $\pi$ y exponerlo en un pdf con \LaTeX{}. 
\end{abstract}

%\thispagestyle{empty}
%++++++++++++++++++++++++++++++++++++++++++++++++++++++++++++++++++++++
\section{Motivaci�n y Objetivos}
 \subsection{rfhy}
 
A lo largo de la historia han sido muchas las formas utilizadas por el 
ser humano para calcular aproximaciones cada vez m�s exactas del n�mero $\pi$.
%
El objetivo de esta pr�ctica de laboratorio es implementar el c�digo \textsf{Python}
que permita a\-pro\-xi\-mar el n�mero $\pi$ con una cierta precisi�n. 
%
$\pi$ se puede calcular mediante integraci�n:

$$\int_{0}^{1} \! \frac{4}{1+x^2}\, dx = 4(atan(1) -atan(0)) = \pi $$

Esta integral se puede aproximar num�ricamente con una f�rmula de cuadratura.
%
Si se utiliza la regla del punto medio se obtiene:

\begin{center}
$ \pi \approx \frac{1}{n} \sum\limits_{i=1}^{n}f(x_i)\,$, 
con $f(x) = \frac{4}{(1+x^2)}\,$,
$x_i = \frac{i - \frac{1}{2}}{n}$,
para $i = 1, \dots, n$
\end{center}

%++++++++++++++++++++++++++++++++++++++++++++++++++++++++++++++++++++++
\section{Ejercicios propuestos}

Escriba un programa que reciba como entrada el n�mero de subintervalos 
con los que se desea abordar la aproximaci�n de $\pi$.
%
A partir de �l se deben calcular y mostrar por la consola:

\begin{enumerate}
  \item
    Algunas aproximaciones hist�ricasde valores de $\pi$, anteriores a la �poca computacional, se muestran en la siguente tabla:\\ 
\begin{table}{}
\begin{center}
\begin{tabular}{|l|r|c|}
Cultura    &  Aproximacion & Error\\ \hline
Egipcia    &  3,1605       & 6016 ppm\\ \hline
Babil�nica &  3,125        & 5282 ppm\\ \hline
India      &  3,09         & 16422 ppm\\ \hline
\end{tabular}
\end{center}
\end {table}
\end{enumerate}

Por ejemplo, si se utilizan 4 subintervalos, la salida deber�a ser: 



%++++++++++++++++++++++++++++++++++++++++++++++++++++++++++++++++++++++
\section{Entregable}
En la tarea habilitada para esta pr�ctica en el Aula Virtual, se subir� 
la direcci�n del repositorio \textit{github} donde se ha almacenado la pr�ctica. 

%++++++++++++++++++++++++++++++++++++++++++++++++++++++++++++++++++++++
\section{Para saber m�s...}

Ampl�e el programa \textsf{Python} que ha desarrollado para que el n�mero de
subintervalos se pueda obtener tambi�n desde la l�nea de comandos.

\begin{thebibliography}{1}
\bibitem{python} Tutorial de Python. http://docs.python.org/2/tutorial/ 
\end{thebibliography}

\end{document}
